\documentclass[13pt, t]{beamer}

% Presento style file
\usepackage{config/presento}

% custom command and packages
% custom packages
\usepackage{textpos}
\setlength{\TPHorizModule}{1cm}
\setlength{\TPVertModule}{1cm}

\newcommand\crule[1][black]{\textcolor{#1}{\rule{2cm}{2cm}}}



% Information
\title{\huge lingphonology}
\subtitle{проект компьютерной программы для автоматического описания фонологии языка (на примере андийского)}
\author[shortname]{Г. А. Мороз}
\institute{Международная лаборатория языковой конвергенции, НИУ ВШЭ}
\date{\begin{center} 
\large 26 февраля 2018 г. \bigskip \\ {\color{colorblue} Фонологический семинар (МГУ)} \end{center}}

\begin{document}

\begin{frame}[plain]
\maketitle
\end{frame}

\begin{frame}{Предыстория}
\begin{itemize}
\item зиловский диалект андийского языка (нахско-дагестанские)
\item несколько полевых поездок в 2016, 2017 годах
\item участники: Неж Рошан, Самира Ферхеес, Александра Мартынова, Айгуль Закирова
\item маленький корпус, небольшой словарь в  FLEx около 2500 корней и аффиксов \pause
\item и вот настало время писать фонологический раздел в статье, посвященной описанию данного идиома \pause
\item вместо того, чтобы скопировать уже готовый текст я решил сделать data-driven описание
\begin{itemize}
\item  выкачал словарь из FLEx в формате .xml
\item переконвертировал его в формат .csv
\item запустил в R программу, которая бы посчитала мне все звуки… \pause
\item …и  обнаружил несколько звуков, которые не было в описании
\end{itemize}
\end{itemize}
\end{frame}

\begin{frame}{Мечты}
\begin{enumerate}
\item Исследователь получает свои данные в каком-то машиночитаемом формате.
\item Программа переводит все из самых разных форматов (простые~.csv и .txt, .TextGrid из Praat, .xml из FLEx, .xml из ELAN, Toolbox и др.) в некоторый свой формат.
\item Программа анализирует данные и хранит где-то промежуточные результаты
\item Программа автоматически составляет фонологический отчет:
\begin{itemize}
\item набор и частотность всех консонантных сегментов
\item набор и частотность всех вокалических сегментов
\item инвентарь супрасегментных единицы
\item описание слоговой структуры, частотность тех или иных слоговых структур
\item фонотактические ограничения (с контекстом разного размера)
\item ограничения уровня слова (гармония и гармонические эффекты, ударения, тоны...)
\end{itemize}
\end{enumerate}
\vfill
{\Large lingphonology} --- пакет для языка R
\end{frame}

\begin{frame}{Собственный формат: IPA}
\begin{itemize}
\item `кузнечик' (зиловский, андийский)
\begin{itemize}
\item ʁat͡sa
\item ʁatsa
\item ʁaca
\item гъацIа
\item гъац1а
\item … \pause
\end{itemize}
\item[→] нужно, чтобы пользователь предоставил соответствия пользовательской транскрипции и IPA \pause
\item[pro] мы получим соответствия единиц в пользовательской нотации с уже разработанной системой фонологических признаков, т. е. мы сможем
\begin{itemize}
\item  отличить гласные от согласных
\item строить модели слога
\item находить нарушение иерархии сонорности
\item …
\end{itemize}
\item[contra] то чего в  IPA нет, придется описывать обходным путем
\end{itemize}
\end{frame}

\begin{frame}{Проблемы IPA}
\begin{itemize}
\item 2 января 2008 Салли Томпсон опубликовала статью в блоге Language Log с заголовком \href{http://itre.cis.upenn.edu/~myl/languagelog/archives/005287.html}{``Why I Don't Love the International Phonetic Alphabet''}
\begin{itemize}
\item графические минусы (неудобно писать, печатать и т. п.)
\item аффрикаты vs. сочетания звуков: польские tʃɨ `три' vs. t͡ʃɨ `вопросительная частица'
\item передний гласный a \pause и его курсивный вариант \textit{a} \pause
\end{itemize}
\item мои собственные претензии
\begin{itemize}
\item отсутствие некоторых звуков, например, свистяще-шипящих
\item многозначность. Угадайте, как описан русский в \href{https://www.cambridge.org/core/services/aop-cambridge-core/content/view/55589EC639ADEF1764B5ECD0B76970FA/S0025100314000395a.pdf/russian.pdf}{журнале IPA}?\\
ʃ vs. ʃʲ(ː) или ʃ vs. ɕ(ː) или ʃˠ vs. ʃʲ(ː) или ʃˠ vs. ɕ(ː) или ʂ vs. ʃ(ː)\\
c и ɟ --- обозначают разное в разных грамматиках
\item причем у нас разные привычки с Максимом Федотовым, Инной Зибер, Митей Николаевым, Андреем Никулиным
\item каждый раз когда я открываю статью IPA журнала о языке, который я знаю, то обнаруживаю что-то странное
\end{itemize}
\end{itemize}
\end{frame}

\begin{frame}{IPA}
\begin{itemize}
\item IPA из базы данных \href{http://phoible.org/}{Phoible}
\begin{itemize}
\item 2160 звуков, затранскрибированные в IPA
\item \href{http://phoible.org/parameters/1F27FEFCD8402C09E00300258B5A7EE2#2/-3.9/75.2}{база данных} (но ей стоит верить крайне осторожно, см. доклад Саши Архипова... сто лет назад)
\end{itemize}
\end{itemize}
\end{frame}

\framecard[colorblue]{{\color{colorwhite} \huge Спасибо за внимание! \bigskip\\
\Large Пишите письма\\
agricolamz@gmail.com}}

\end{document}