\documentclass[13pt, t]{beamer}

% Presento style file
\usepackage{config/presento}
\usepackage{forest}
% custom command and packages
\input{config/custom-command}

% Information
\title{\huge lingphonology}
\subtitle{проект компьютерной программы для автоматического описания фонологии языка (на примере андийского)}
\author[shortname]{Г. А. Мороз}
\institute{Международная лаборатория языковой конвергенции, НИУ ВШЭ}
\date{\begin{center} 
\large 26 февраля 2018 г. \bigskip \\ {\color{colorblue} Фонологический семинар (МГУ)} \end{center}}

\begin{document}

\begin{frame}[plain]
\maketitle
\end{frame}

\begin{frame}{Предыстория}
\begin{itemize}
\item зиловский диалект андийского языка (нахско-дагестанские)
\item несколько полевых поездок в 2016, 2017 годах
\item участники: Неж Рошан, Самира Ферхеес, Александра Мартынова, Айгуль Закирова
\item маленький корпус, небольшой словарь в  FLEx около 2500 корней и аффиксов \pause
\item и вот настало время писать фонологический раздел в статье, посвященной описанию данного идиома \pause
\item вместо того, чтобы скопировать уже готовый текст я решил сделать data-driven описание
\begin{itemize}
\item  выкачал словарь из FLEx в формате .xml
\item переконвертировал его в формат .csv
\item запустил в R программу, которая бы посчитала мне все звуки… \pause
\item …и  обнаружил несколько звуков, которые не было в описании
\end{itemize}
\end{itemize}
\end{frame}

\begin{frame}{Мечты}
\begin{enumerate}
\item Исследователь получает свои данные в каком-то машиночитаемом формате.
\item Программа переводит все из самых разных форматов (простые~.csv и .txt, .TextGrid из Praat, .xml из FLEx, .xml из ELAN, Toolbox и др.) в некоторый свой формат.
\item Программа анализирует данные и хранит где-то промежуточные результаты
\item Программа автоматически составляет фонологический отчет:
\begin{itemize}
\item набор и частотность всех консонантных сегментов
\item набор и частотность всех вокалических сегментов
\item инвентарь супрасегментных единицы
\item описание слоговой структуры, частотность тех или иных слоговых структур
\item фонотактические ограничения (с контекстом разного размера)
\item ограничения уровня слова (гармония и гармонические эффекты, ударения, тоны...)
\end{itemize}
\end{enumerate}
\vfill
{\Large lingphonology} --- пакет для языка R
\end{frame}

\begin{frame}{Общий формат}
\begin{forest}
[
{\begin{tabular}{|l|l|l|l|l|l|l|}
\hline
text id & id & word & translation & … & POS & … \\ \hline
… & … & … & … & … & … & …\\ \hline
4 & 231 &  ʁats'a & кузнечик & … & N & … \\ \hline
… & … & … & … & … & …& …\\ \hline
\end{tabular}}
, for tree = {edge=<-,  grow' = west}
	[тексты .TextGrid]
	[словарь .csv]
	[тексты .txt]
	[тексты Toolbox]
	[словарь Toolbox]
	[тексты .eaf]
	[тексты FLEx]
	[словарь FLEx]]
\end{forest}
\end{frame}


\begin{frame}{Конвертер из пользовательской транскрипции в IPA}
\begin{itemize}
\item `кузнечик' (зиловский, андийский)
\begin{itemize}
\item ʁat͡sa
\item ʁatsa
\item ʁaca
\item гъацIа
\item гъац1а
\item … \pause
\end{itemize}
\item[→] нужно, чтобы пользователь предоставил соответствия пользовательской транскрипции и IPA \pause
\item[pro] мы получим соответствия единиц в пользовательской нотации с уже разработанной системой фонологических признаков, т. е. мы сможем
\begin{itemize}
\item  отличить гласные от согласных
\item строить модели слога
\item находить нарушение иерархии сонорности
\item …
\end{itemize}
\item[contra] то чего в  IPA нет, придется описывать обходным путем
\end{itemize}
\end{frame}

\begin{frame}{Конвертер из пользовательской транскрипции в IPA}
\begin{itemize}
\item 2 января 2008 Салли Томпсон опубликовала статью в блоге Language Log с заголовком \href{http://itre.cis.upenn.edu/~myl/languagelog/archives/005287.html}{``Why I Don't Love the International Phonetic Alphabet''}
\begin{itemize}
\item графические минусы (неудобно писать, печатать и т. п.)
\item аффрикаты vs. сочетания звуков: польские {\Large tʃɨ} `три' vs. {\Large t͡ʃɨ} `вопросительная частица'
\item передний гласный {\Large a} \pause и его курсивный вариант {\Large \textit{a}} \pause
\end{itemize}
\item мои собственные претензии
\begin{itemize}
\item отсутствие некоторых звуков, например, свистяще-шипящих
\item многозначность. Угадайте, как описан русский в \href{https://www.cambridge.org/core/services/aop-cambridge-core/content/view/55589EC639ADEF1764B5ECD0B76970FA/S0025100314000395a.pdf/russian.pdf}{журнале IPA}?\\
{\Large ʃ} vs. {\Large ʃʲ(ː)} или {\Large ʃ} vs. {\Large ɕ(ː)} или {\Large ʃˠ} vs. {\Large ʃʲ(ː)} или или {\Large ʂ} vs.{\Large  ʃ(ː)}\\
{\Large c} и {\Large ɟ} --- обозначают разное в разных грамматиках
\item причем у нас разные привычки с Максимом Федотовым, Инной Зибер, Митей Николаевым, Андреем Никулиным
\item каждый раз когда я открываю статью IPA журнала о языке, который я знаю, то обнаруживаю что-то странное
\end{itemize}
\end{itemize}
\end{frame}

\begin{frame}{Конвертер из пользовательской транскрипции в IPA}
\begin{itemize}
\item IPA из базы данных \href{http://phoible.org/}{Phoible}
\begin{itemize}
\item 2160 звуков 1672 языков, затранскрибированные в IPA
\item \href{http://phoible.org/parameters/1AD6D96C35E6ADA62F1EAF5B167F75F7}{база данных} (но ей стоит верить крайне осторожно, см. доклад Саши Архипова... сто лет назад)
\item SPE признаки для каждого звука
\end{itemize}
\item но тут же возникли проблемы: \pause
\begin{itemize}
\item в базе данных не оказалось  {\Large dʒ}\pause, потому что там {\Large d̠ʒ} \pause
\item в базе данных не оказалось  {\Large pʼʷ}\pause, потому что там {\Large pʷʼ} \pause
\item в базе данных не оказалось {\Large çʷː}\pause. Просто не оказалось.
\end{itemize}
\item кроме соответствия некоторому изводу IPA, конвертер нужно проверять на омографию
\item омографии можно избежать, если научиться обрабатывать графемные контексты (<е> --- в начале слова vs. в середине слова в русском) \pause
\item должны быть возможно исключить какие-то фрагменты (например, заимствования, идиофоны)
\end{itemize}
\end{frame}

\begin{frame}{Общий формат → данные}
\begin{itemize}
\item инвентари и частотность
\item фонотактика
\begin{itemize}
\item смотрим на все консонантные кластеры
\item смотрим на все вокалические сочетания
\item смотрим на (квази)гармонические эффекты
\item смотрим на позицию ударения
\item смотрим на поведение тонов
\end{itemize}
\item слоговая структура
\begin{enumerate}
\item смотрим на все инициали
\item смотрим на все финали
\item моделируем общую схему слога
\item применяем модель ко всем слогам и смотрим, где она ломается
\end{enumerate}
\end{itemize}
\end{frame}

\begin{frame}{Заключение}
\begin{itemize}
\item в целом эта система позволит создавать data-driven фонологические описания
\item это в свою очередь позволит ввести меру доверия фонологического описания: описанию, основанном на списке из 700 слов, доверия меньше, чем описанию из 5000 слов
\item проверять исследовательские данные --- в случае, если какие-то единицы есть в словаре или тексте, но нет в конвертере
\item garbage in --- garbage out: если кто-то не различил в своих данных дентальные и альвеолярные t, то и в фонологический отчет это не попадет
\end{itemize}
\end{frame}

\framecard[colorblue]{{\color{colorwhite} \huge Спасибо за внимание! \bigskip\\
\Large Пишите письма\\
agricolamz@gmail.com}}

\end{document}